\documentclass{article}
\usepackage[utf8]{inputenc}
\usepackage{a4wide}
\usepackage{amssymb}
\usepackage{todonotes}
\usepackage[
    colorlinks = true,
    linkcolor = blue,        
    urlcolor  = violet,
]{hyperref}

\title{\textbf{Application of self-organizing map to the analysis of the geotechnical data}.\\ \textit{A project overview}} 

\author{%
\begin{tabular}{c} Łukasz Lepak \\ The Institute of Computer Science \\ Artificial Intelligence Division \\ lepak.dokt@pw.edu.pl \end{tabular} \and
\begin{tabular}{c} Eryk Warchulski \\  The Institute of Computer Science \\ Artificial Intelligence Division \\ eryk.warchulski.dokt@pw.edu.pl \end{tabular} 
}
\date{}
\begin{document}

\maketitle

\section{Project aim}
The aim of the project is to construct and evaluate a self-organizing map model (SOM)\footnote{T. Kohonen, M. R. Schroeder, T. S. Huang, “Self-Organizing Maps”, Springer-Verlag, 2001.} to given geotechnical data in the context of unsupervised learning tasks.
Given data contains 8 files with a three-dimensional data frame. Each row of the data frame has the structure written below:
\begin{equation}
  \left(d, p_1, p_2 \right) \in \mathbb{R}^{3}
\end{equation}

\noindent where $d$ stands for depth of stuck sensor in the ground and $p_i, i\in\left\{1, 2\right\}$ are sensor readings. Analysis of this data will be included in the final report.\\
For better reference, we will also use a simpler method (e.g. PCA) as a base method. It will help us to decide whether such a complex model like SOM has any added value to the performed analysis. 

\section{Technical issues}

We are considering to use simultaneously two different programming languages to accomplish our project goals: \texttt{Python 3.9.0} and \texttt{R 4.0.3} programming language.
The first technology is very suitable for tasks like constructing, learning and evaluating machine learning models. This property follows from efficient implementations of machine learning libraries in low-level languages like \texttt{C++} and expressive FFI (Foreign Function Interface) between them and Python. Also, there is plenty of SOM libraries written in \texttt{Python} -- some of them are listed below:

\begin{itemize}
  \item \texttt{som} \footnote{\url{https://pypi.org/project/som/}}
  \item \texttt{som-learn} \footnote{\url{https://pypi.org/project/som-learn/}}
  \item \texttt{kohonen}. \footnote{\url{https://pypi.org/project/kohonen/}}
\end{itemize}

We are not sure yet which would fit our demands. There is also the possibility that (in case of lack of proper implementation) we will write SOM from scratch with \texttt{Tensorflow} or \texttt{Theano} frameworks.
The second technology, \texttt{R} programming language, due to excellent libraries for data processing, transformation and visualization will be involved in tasks such as data preparation and simple exploratory data analysis.

\end{document}


